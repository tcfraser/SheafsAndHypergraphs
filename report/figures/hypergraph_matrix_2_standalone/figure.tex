\documentclass[preview]{standalone}

\usepackage{geometry}
\geometry{paperwidth=4in, paperheight=2in}

% === To allow for labeling of rows and columns of a matrix ===
\usepackage{../../packages/kbordermatrix}

\usepackage{xcolor}
\usepackage{ifthen}
\newcommand{\nf}[1]{\ifthenelse{\equal{#1}{0}}{\textcolor{gray}{0}}{\mathbf{1}}}

\begin{document}
$
\kbordermatrix{
    & \mathcal{E}_1 & \mathcal{E}_2 & \mathcal{E}_3 & \mathcal{E}_4 & \mathcal{E}_5 & \mathcal{E}_6 \\
    n_1 & \nf{1} & \nf{0} & \nf{0} & \nf{1} & \nf{0} & \nf{0} \\
    n_2 & \nf{0} & \nf{1} & \nf{1} & \nf{1} & \nf{1} & \nf{0} \\
    n_3 & \nf{1} & \nf{0} & \nf{0} & \nf{0} & \nf{0} & \nf{1} \\
    n_4 & \nf{1} & \nf{0} & \nf{0} & \nf{0} & \nf{1} & \nf{0} \\
    n_5 & \nf{1} & \nf{1} & \nf{0} & \nf{0} & \nf{1} & \nf{1} \\
    n_6 & \nf{0} & \nf{1} & \nf{1} & \nf{1} & \nf{0} & \nf{1} \\
    n_7 & \nf{0} & \nf{0} & \nf{1} & \nf{0} & \nf{1} & \nf{0} \\
}
$
\end{document}
