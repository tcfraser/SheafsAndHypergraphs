\documentclass[preview]{standalone}

\usepackage{geometry}
\geometry{paperwidth=4in, paperheight=2in}

% === To allow for labeling of rows and columns of a matrix ===
\usepackage{../../packages/kbordermatrix}
\renewcommand{\kbldelim}{(} % Left delimiter
\renewcommand{\kbrdelim}{)} % Right delimiter

\usepackage{xcolor}
\newcommand{\kone}{\mathbf{1}}
\newcommand{\kzer}{\textcolor{gray}{0}}

\begin{document}
$
\kbordermatrix{
    & \textcolor{blue}{\mathcal{E}_1} & \textcolor{red}{\mathcal{E}_2} & \textcolor{green}{\mathcal{E}_3} & \textcolor{yellow}{\mathcal{E}_4} & \textcolor{orange}{\mathcal{E}_5} \\
    n_1 & \kone & \kzer & \kone & \kzer & \kzer \\
    n_2 & \kzer & \kzer & \kone & \kzer & \kone \\
    n_3 & \kone & \kzer & \kzer & \kone & \kzer \\
    n_4 & \kone & \kzer & \kzer & \kzer & \kone \\
    n_5 & \kone & \kone & \kzer & \kzer & \kzer \\
    n_6 & \kzer & \kone & \kzer & \kone & \kzer \\
    n_7 & \kzer & \kone & \kzer & \kzer & \kzer \\
}
$
\end{document}
